\documentclass[11pt]{article}
\usepackage{mypackages}
\begin{document}

\section{Results}

For each of the experiments discussed in the previous section,
we will be using \textit{timesteps} as a time unit.
In both Cartpole and Atari games a timestep corresponds to
taking an action, however it is defined for the game.
In CartPole an action is taken in each state, but in the Atari games
there the action is repeated four times.

\subsection{Actor-Critic with eligibility traces}

In figure \ref{fig:cp_et} the results of our experiments using the Actor-Critic method with eligibility
traces are shown.
The experiments were run five consecutive times on an Intel(R) Xeon(R) CPU E5-2670 v2 @ 2.50GHz\cite{intel}, for $200.000$ timesteps each.


% Template
\begin{figure}[H]
    \begin{subfigure}{.5\textwidth}
        \centering
        \includegraphics[scale=0.25]{include/template.jpg}
        \caption{The score of the Actor-Critic method with eligibility
            traces as a function of the number of timesteps
            the algorithm have been running.}
    \end{subfigure}
    \begin{subfigure}{.5\textwidth}
        \centering
        \includegraphics[scale=0.25]{include/template.jpg}
        \caption{The score of the Actor-Critic method with eligibility
            traces as a function of the real-time the algorithm
            have been running.}
    \end{subfigure}
    \caption{Average results of five runs of the Actor-Critic method
        with eligibility traces. The [indsæt farve] is the plot of the
        raw results and the [indsæt farve] is a plot that describes the
        mean of the the [K] surrounding data points.} 
     \label{fig:cp_et}
\end{figure}

Skriv noget om hvad vi ser.

\subsection{A3C - Cartpole}

For this method we also ran our experiments for $200.000$ timesteps
five consecutive times.
We have run experiments for five different thread settings - 16, 8, 4, 2 and 1.
Each of the experiments had access to one Intel(R) Xeon(R) CPU E5-2670 v2 @ 2.50GHz for each thread it spawned.
Figure \ref{fig:a3c_cp_all} shows the average result of the experiments.
The graphs are smoothed such that each data point is equivalent to
the mean of the [K] surrounding points.

\begin{figure}[H]
    \begin{subfigure}{.5\textwidth}
        \centering
        \includegraphics[scale=0.25]{include/template.jpg}
        \caption{The score of the A3C method as a function
        of the number of timesteps the algorithm have been running for
        all amounts of threads.}
    \end{subfigure}
    \begin{subfigure}{.5\textwidth}
        \centering
        \includegraphics[scale=0.25]{include/template.jpg}
        \caption{The score of the A3C method as a function
        of the real-time the algorithm have been running for
        all amounts of threads}
    \end{subfigure}
    \caption{Average results of five runs of A3C method on the CartPole
        game for each thread setting. The plots describe the mean of
        the surrounding [K] points.}
     \label{fig:a3c_cp_all}
\end{figure}

Skriv noget om resultatet.



\end{document}
