\documentclass[11pt]{article}
\usepackage{mypackages}
\begin{document}

\section{Conclusion}

\begin{itemize}
    \item[--] Hvad var målet med projektet
    \item[--] Hvad skete der i projektet?
    \item[--] Resultater
    \item[--] Pointe om balance mellem tid brugt på interaction og opdateringer
    \item[--] I fremtiden kunne det være fedt at prøve at finde ud af hvor mange agenter man kan bruge og stadig beholde stabiliteten.
\end{itemize}


The aim of this project was to investigate the effects of asynchronous
training in Reinforcement Learning.
A basic Actor-Critic method was presented, extended by the use of eligibility traces,
as well as the more recent Asynchronous Advantage Actor-Critic algorithm.
We were able to solve the Cartpole problem and some of the games from the
Atari 2600 system, and show that it was possible to obtain a
speed-up of 697\% by using 16 learning agents in parallel, compared to
using only a single thread, in the games from the Atari 2600 system.
However, we only saw a speed-up of 3,4\% in the time it took for the CartPole problem
to complete 200.000 timesteps.
This indicates [pointe om balance].

\end{document}
