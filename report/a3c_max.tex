\documentclass[11pt]{article}
\usepackage{mypackages}
\begin{document}

% Intro til A3C

\section{Asynchronous Advantage Actor-Critic}

A problem with the Actor-Critic method is that training the model can become very time consuming,
when it is solving challenging tasks such as Atari games.
To speed up the training process it is possible to use 
multiple Actor-Critic models to 
solve the problem locally in parallel.
These models share a set of parameters, which they update asynchronously,
and thus, each model can benefit from the experience gained by the other models.

% Uformel forklaring
Each Actor-Critic model interacts with its own enviornment
wich means the training becomes more diverse, since it is independent of the other
environments.
Diverse experience stabilizes the training process,
since more situations will be encountered,
while the local models have approximately the same parameters.
The major challenge of asynchronous training is therefore to decide how
the local models can influence the global paramaters in a healthy
way.

The approach presented in \cite{a3c} use several local models
and a single global model, whose only purpose is to keep track of the global parameters,
to train asynchronously.
Each of the local models use a copy of the global parameters to run through a problem
and update the global parameters based on the local experience.
% Global updating scheme
The global update should be performed such that
the elegibility of each component determines the size of the
gradient step.
In this project we use \textit{Root Mean Square propagation} (RMSprop)
to perform the asynchronous update of the global parameters.
RMSprop is an approach to \textit{gradient descent} 
that uses the magnitude of the recent gradients to
determine the step-size of the update.
The update is given by 
\begin{equation}
    \theta = \theta - \eta \frac{\Delta \theta}{\sqrt{g + \epsilon}}
\end{equation}
where $\eta$ is the learning rate of the gradient descent, and 
\begin{equation}
    g = \alpha g + (1 - \alpha) \Delta \theta^2
\end{equation}
Here $g$ is the magnitude of the gradient and $\alpha$ works as a decay rate, determining how much each new update
can influence the magnitude and how much the previous magnitude fades away.
In this project $g$ is shared between all local models, which has a stabilizing
effect, since “bad” gradient steps won't have as much influence as they would,
if each local model had its own $g$.

We want to update the weights of the global model
when a significant change has occured.
To decrease the amount of insignificant updates we accumulate the gradients
in each of the local models, a number of steps, before performing the asynchronous update.

% Advantage
In section \ref{actor_critic} we used the one-step TD-error to determine
whether an action was good or bad and updated the model accordingly, in order to 
reinforce good behaviour and likewise discourage
the model from taking actions that have led to a bad result in the past.
If we used the one-step TD-error, we would be performing
a lot of updates where no significant changes has happened, so
instead we use the \textit{advantage} gained from taking an action.
The advantage is computed after performing an action by comparing the experienced return the
expected return.
The advantage is given by
\begin{equation}
    A = R - \hat{v}(s, \mathbf{w}')
\end{equation}
where $R$ is the experienced return that followed from taking
a certain action, and $\hat{v}(s, \mathbf{w}')$ 
is the estimate of the expected return from state $s$.

As mentioned previously, we want to accumulate the directional
steps we should take, to increase the chance of encountering a significant change.
This accumulation resembles the gradient ascent from \ref{eq:ac_theta} and
is given by
\begin{equation}
    d\theta' = d\theta' + \nabla_{\theta'} \log \pi(a|s, \theta')(R - \hat{v}(s, \mathbf{w}'))
\end{equation}
The key difference lie in the usage of the advantage.
As described in section \ref{sec:actor_critic} taking the gradient of $\log \pi(a|s, \theta')$
w.r.t. to $\theta'$ is an attempt to normalize the influence of all actions, such that the
gradient steps of actions that are rarely picked should be weighted the same
as those belonging to actions that are picked more often.
The advantage of taking action $a$ determines in which direction the gradient step
is taken resulting in actions that perform poorly being less likely to be sampled
and vice versa.

In the previous sections we updated the value function proportionally to the
one-step TD-error.
Since we are using the n-step return, $R$, we can use the advantage, $R - \hat{v}(s, \mathbf{w'})$,
as an error measure.
We want the expected return of a state to be as close to the experienced return as possible,
so to emphasize this, we use the square advantage to express the magnitude of the
gradient step that should be taken in the asynchronous update.
In \cite{a3c} this update is specified as
\begin{equation}
    d\mathbf{w}' = d\mathbf{w}' + \frac{\partial (R - \hat{v}(s, \mathbf{w'}))^2}{\partial \mathbf{w}'}
\end{equation}
which is the same as taking the gradient w.r.t. $\mathbf{w}'$ since
\begin{equation*}
    \begin{aligned}
        \frac{\partial (R - \hat{v}(s, \mathbf{w}'))^2}{\partial \mathbf{w}'} & = \bigg(
            \frac{\partial (R - \hat{v}(s, \mathbf{w}'))^2}{\partial \mathbf{w}'_1},
            \frac{\partial (R - \hat{v}(s, \mathbf{w}'))^2}{\partial \mathbf{w}'_2}, \dots,
            \frac{\partial (R - \hat{v}(s, \mathbf{w}'))^2}{\partial \mathbf{w}'_d} \bigg) \\
            & = \nabla_{\mathbf{w}'} (R - \hat{v}(s, \mathbf{w}'))^2
    \end{aligned}
\end{equation*}
where $\mathbf{w}'_1, \mathbf{w}'_2, \dots, \mathbf{w}'_d$ are the components of $\mathbf{w}'$.
In other words whenever the advantage is large, we want the gradient step to be big as well.

As an extra measure to favor exploration, an \textit{entropy} has been added
to the policy approximator of the local models.
The entropy of the policy corresponds to the spread of action probabilities.
This means that the entropy will be large when the policy is close to being
uniform and small when the probabilities approach determinism.
In short, the entropy is used to discourage premature convergence
by encouraging the model to be more conservative about how sure it is
of which action is correct.
Using the entropy, $H(\pi(s, \theta'))$, alters the accumulation of the gradients
of the policy, such that
\begin{equation}
    \begin{aliegned}
        d\theta' & = d\theta' + \nabla_{\theta'} \log \pi(a|s, \theta')(R - \hat{v}(s, \mathbf{w}')) + \beta \nabla_{\theta'} H(\pi(s, \theta'))\\
                 & = d\theta' + \nabla_{\theta'} (\log \pi(a|s, \theta')(R - \hat{v}(s, \mathbf{w}')) + \beta H(\pi(s, \theta')))
    \end{aliegned}
\end{equation}

The advantages of the A3C algorithm compared to the regular Actor-Critic method
are increased stability and less time consumed by training.
By running multiple Actor-Critic models in parallel, more paths through the environment are explored
at the same time.
All local models try to maximize the return from the state that they're in, which means that
the global parameters are influenced by a lot of different factors at the same time.
The many different updates increase the stability of the training by having
extreme gradient updates cancel each other out.
Using more than one model also decreases the time spent training, since
more state-action transitions can be experienced at the same time.

\end{document}
