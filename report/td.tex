\documentclass[11pt]{article}
\usepackage{mypackages}
\begin{document}

\maketitle

\subsection{Temporal Difference Learning}\label{sec:td}

We now have a way to describe the value of a state, but we seldomly know the
value function $v_\pi$ before we begin solving a problem.
Even with the knowledge of the entire state space and all available actions
it is difficult to compute the true value function.
Instead it is possible to construct an estimator $V(S_t)$ that
approximates $v_\pi(S_t)$.

When a learning agent solves a problem, it gains a lot of experience
about the states it encounters and rewards it earns through different transitions.
Using this experience it is possible to construct an updating scheme
for estimating the approximator,
\begin{equation}
    V(S_t) = V(S_t) + \alpha [G_t - V(S_t)]
\end{equation}
where $G_t$ is the experienced actual return following from state
$S_t$ and $0 \leq \alpha \leq 1$ is a step-size parameter, 
which defines how much each update is able to change the estimate
of the value in state $S_t$.

This estimator only allows us to update the estimate, when an epsiode
has ended, since this is the time at which $G_t$ is known.
Updating this way is inefficient because it would take the estimator
a lot of episodes to converge towards the true state-value function,
due to it being updated seldomly. 

Instead of updating the estimator $V(S_t)$ when an episode has
ended, we can choose a different approach which updates the $V(S_t)$
every time a new state is reached.
From equation \ref{eq:sv} we know that the value function is partly based on its future
estimate, which we can extend to form an update scheme,
\begin{equation}
    V(S_t) = V(S_t) + \alpha  [R_{t+1} + \gamma  V(S_{t+1}) - V(S_t)]
\end{equation}
where we only have to take a single step into the future before we are
able to update the estimate.
Updating the value estimator based on future results is the basis
of \textit{temporal difference learning}(TD) and the quantity $R_{t+1} + \gamma  V(S_{t+1}) - V(S_t)$
is defined as the \textit{one-step TD-error}.
This error describes the difference between the current estimator
and the estimate that is based on actual current experience.

Generally we want to minimize the TD-error, as the function $V(S_t)$
would then be as close to $v_\pi(S_t)$ as possible.
In particular if this error was 0 for all states, then
$V(S_t) = v_\pi(S_t)$ and
\begin{align}
   V(S_t)  & = V(S_t) + \alpha  [R_{t+1} + \gamma  V(S_{t+1}) - V(S_t)]   \\
           & = V(S_t) + \alpha  [V(S_t) - V(S_t)]  \\
           & = V(S_t) 
\end{align}

\subsubsection{Multi-step TD learning}

The benefit of this kind of updating scheme is that we are able to base
part of our estimate on an already existing estimate.
Sometimes it is useful to look at every state-change, but sometimes
changes over a single time step can be small and maybe even insignificant.
The one-step update scheme used a single experienced state transition to estimate
$G_t$ using the reward earned in the transition and the estimate of the
next state.
Basically we want to extend this way of estimating $G_t$ to take $n$ experienced rewards
into account, which is defined as the \textit{n-step} return,
\begin{equation}
    G^{(n)}_t = R_{t+1} + \gamma R_{t+2} + \dots 0 \gamma^n V(S_{t+n})
\end{equation}
where $n \geq 1$, $0 < t \leq T - n$ and $T - n$ is the nth time step before the terminal state
is reached.

Using the n-step return allows the learning agent to base its update on the next $n$ experienced
rewards, which means it is more likely to experience a significant change than the one-step return.
This way of updating the estimate is espicially beneficial in tasks that only sometimes returns a reward
- e.g. in a lot of games a score is only given to the player when an objective is completed.
We want to find these significant changes because the entire chain of estimates 
will be affected, making the estimate converge faster towards the real value function.

\subsubsection{Eligibility traces}

% Intro til parameterized functions
So far we have described the state-value function as a mapping from states to values.
A problem with this approach is that it becomes infeasable to maintain this mapping
when the state space is continuous, since there is an infinite number of states in this case.
To make the transition from a discrete to a continous state space we can use a
parameterized function $\hat{v}(s, \mathbf{w})$ with weights $\mathbf{w}$
to estimate $v_\pi(s)$.
Thus, instead of keeping track of the last $n$ steps, the estimator $\hat{v}(s, \mathbf{w})$
only needs to keep track of the weight vector $\mathbf{w}$, which can be seen as a long-term memory.
There are many ways to construct $\hat{v}(s, \mathbf{w})$, but in this project we will only
be using \textit{deep neural networks} as function approximators,
which will be discussed in detail in section \ref{sec:deep_learning}.

Now, before we could estimate the value function based on the n-step return, but
since the value function now depends on the weights $\mathbf{w}$ we need to find a way
to measure how much each component of the weight vector is influencing the
current estimate.
To do so we can use an \textit{eligibility trace}, $\mathbf{e} \in \R^n$, that
has exactly as many components as the weight vector.

Generally we want to use the elegibility trace to improve the value function by
changing its weights $\mathbf{w}$ based on \textit{gradient ascent}.
The eligibility trace describes the elegibility of its corresponding component in $\mathbf{w}$
- in other words which direction to update the weight to reach a maximum for $\hat{v}(s, \mathbf{w})$.
Thus an elegibility trace for the value estimator can be updated as
\begin{equation}
    \mathbf{e}^\mathbf{w} = \lambda \mathbf{e}^\mathbf{w} + \nabla_\mathbf{w} \hat{v}(s, \mathbf{w})
\end{equation}
where $\lambda$ is a discounting factor used to decrease the elegibility of a component
over time, as it gets closer to a maximum.
Now, the trace describes the trend of the most recent gradients of the value function
w.r.t $\mathbf{w}$, but we also want to base the update of the weights on their
performance.
Since we want the estimate of the value function to improve as we
experience the effect of our actions we genereally want to
update the weights proportionally to the one-step TD-error introduced in
section \ref{sec:td}.
A low TD-error means that we don't have to change much as the value estimator is
close to a maximum, while a high error means that we need to take a larger step
in the directions from the elegibility trace as we are still far from the maximum.
An update based on this approach would look like this
\begin{equation}
    \mathbf{w} = \mathbf{w} + \eta \delta \nabla e^\mathbf{w}
\end{equation}
where $\eta$ determines how much the weights can change each step.



\end{document}
