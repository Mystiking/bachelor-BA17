\documentclass[11pt]{article}
\usepackage{mypackages}
\begin{document}

\maketitle

\subsection{Estimating the value of actions}

We have already introduced the goal of the learning agent as being the
task of maximising the total reward.
This we can formally define as maximizing the \textit{return}, $G_t$, as
a function of the sequence of rewards following time step $t$ - $R_{t+1}$,
$R_{t+2}$, $R_{t+3}$, $\hdots$.
In its simplest form the return can be expressed as the sum of the
rewards
\begin{equation}\label{G_t}
    G_t \coloneqq R_{t+1} + R_{t+2} + \hdots + R_{T} = \sum\limits_{i}^T R_{t+i}
\end{equation}
where $T$ is the point in time where the \textit{terminal state} is reached
\cite{RLBook}.

This definition makes sense in tasks that have a natural ending.
The terminal state is in this case the state from where no
other states can be reached.
However, if there is no apparent terminal state, equation \ref{G_t}
becomes problematic as the final timestep $T$ is $\infty$ which means $G_t = \infty$.

To address this problem we introduce a way to weigh the rewards by how far into the
future they lie - a concept called \textit{discounting}.
The basic idea behind discounting is that rewards recieved $k$ steps into the future
is worth $\gamma^k$ less than a reward recieved at present time, where
$\gamma$ is the \textit{discount rate} and $0 \leq \gamma \leq 1$.

Using the discount we can redefine the return $G_t$ as the sum of the discounted
rewards given by
\begin{equation}\label{gammaG_t}
    G_t \coloneqq R_t + \gamma * R_{t+1} + \gamma^2 * R_{t+2} + \hdots 
        = \sum\limits_{k=0}^\infty \gamma^k * R_{t+k+1}
\end{equation}
with $\gamma$ close to 1 weighting future rewards strongly and $\gamma$ close to 0
putting more emphasis on immediate rewards.


Now that the return has been defined, consider the response from the environment at time $t + 1$
to the action taken at time $t$.
According to the agent-environment model (Fig. \ref{agent_environment}) this response consists
of a new state $S_{t+1} = s'$ and a corresponding reward $R_{t+1} = r$.
Assuming the response depends on all past actions $A_{0}, A_{1}, \cdots, A_{t-1}$ and
corresponding responses $S_{1}, R_{1}, \cdots, S_{t}, R_{t}$, the probability
of arriving in state $S_{t+1}$ and receiving reward $R_{t+1}$ can be defined as the
joint probability given by
\begin{equation}
    \mathds{P}(S_{t+1} = s', R_{t+1} = r | S_0, A_0, R_1, \cdots, S_{t-1}, A_{t-1}, R_{t}, S_{t}, A_{t})
\end{equation}
where $S_0$ is the inital state and $A_0$ the first action taken.

Before we move on to discuss how to estimate the value of taking an action in a state, as
well as estimating the value of a state, we an assumption about the reinforcement
learning tasks we will encounter.
The assumption is that the agent-environment model has the \textit{Markov property}.
This means that the state signal is able to retain all relevant information
from the past, or in other words, that the state signal is able to summarize
everything important about the complete sequence of states leading to it.
Having made this assumption the joint probability can now be written as
\begin{equation}
    \mathds{P}(S_{t+1} = s', R_{t+1} = r | S_t, A_t)
\end{equation}
When the agent-environement model has the Markov property the reinforcement
learning task is called a \textit{Markov decision process (MDP)}

From equation \ref{gammaG_t} we know that the discounted return
is defined as $G_t = \sum\limits_{k=0}^\infty \gamma^k * R_{t+k+1}$.
We have also defined a policy $\pi$ as a mapping from state $s$ and action $a$
to the probability $\pi(a, s)$ of taking action $a$ in state $s$ for all states
and all actions available in the corresponding states.
The value of the state $s$ can thus be described as the \textit{expected return}
when starting in state $s$ and following policy $\pi$ thereafter\cite{RLBook}.
For MDPs we are thus able to define the value of state $s$ as
\begin{equation}
    v_\pi(s) = \mathds{E}[G_t | S_t = s] = \mathds{E}\big[\sum\limits_{k=0}^\infty \gamma^k * R_{t+k+1} \big| S_t = s\big]
\end{equation}
because the state $s$ contains all relevant information of the states leading up to it. We call the function $v_{\pi}$ the \textit{state value function}. Now we mathematical can define how good it's to be in a given state, and know we going to the problem of the value for taking a action in a given state. Similar to the \textit{state value function} we define a \textit{action value function}, which is the expected return when starting at state $s$ taking action $a$ and following policy $\pi$ thereafter. For a MDP algorithm which follow the Markov property the action value function is defined as
\begin{equation}
    q_{\pi}(s, a) = \mathds{E}[G_{t} | S_{t} = s, A_{t} = a] = \mathds{E}\bigg[\sum_{k = 0}^{\infty} \gamma^{k} R_{t + k + 1} \bigg| S_{t} = s, A_{t} = a \bigg]
\end{equation}
Now we have defined both a action and value function which can be used for learn about the environment, but we are using $G_{t}$ the actual return for a episode. So we can only update the state and action value function after a ended episode. In section (REF TIL TD) we will discuss how we can update the state and value function after each time step.



\subsection{Policy improvement/update/iteration}

The reason that we using a policy for computing the value function is for finding better policies. When using policy $\pi$ in a state, we already know the expected return by taking action $a$, but could we improve the return by chancing policy? The simplest way for improving a policy is by taking an action in the state

%\printbibliography
%\bibliography{citations}
%\bibliographystyle{plain}
\end{document}