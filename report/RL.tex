\documentclass[11pt]{article}
\usepackage{mypackages}
\begin{document}

\maketitle

\section{Reinforcement Learning}

Reinforcement Learning is a machine learning method that focus on learning from interaction.
The basis of reinforcement learning is that given a set of states $S$, that describes the environment,
and a set of actions $A$, that describes how we can interact with the environment,
a \textit{policy} is derived.
A policy describes which action $a$ to take in which state $s$.
In order to determine wether a policy is good or bad, reinforcement learning uses
rewards and punishments.
Determining to reward or punish the algorithm depends on the problem.
If the problem is simple it is often easier to give rewards, or punishments, based on actions
while it might be better to give more difficult problems only a reward at the end of an \textit{episode}.

An episode starts when the environment is initialised and ends when a terminal state is reached.
For some problems it is difficult to determine if there is a terminal state, since there
is no natural end.

In order to develop a good policy reinforcement learning methods use \textit{exploration} and \textit{exploitation}.

Contrary to supervised learning, where a classifier is build from a set of labeled input and output data,
reinforcement learning focus on i mproving by \textit{exploration} and \textit{exploitation}.

In order to learn how to solve a given problem reinforcement learning needs to be able to
adapt to previously unseen circumstances.
It does this by 

\documentclass[11pt]{article}
\usepackage{mypackages}
\begin{document}

\maketitle

\subsection{State-value functions}

- The notion of "how good is it to be in this state"
- Defined with respects to policies
- MDP $\rightarrow$ $v_\pi(s)$
- Satisfies a recursive relatonship between current state and "all" successor states
- Optimal value functions?


%\printbibliography
%\bibliography{citations}
%\bibliographystyle{plain}
\end{document}

\documentclass[11pt]{article}
\usepackage{mypackages}
\begin{document}

\maketitle

\section{Value Functions}

Most reinforcement learning algorithms estimates a value function, that compute how profitable it is to be in a specific state, or take a specific action given the state. The term \textit{"profitable"} means what the expected future reward will become. Generally in reinforcement learning we have two value function \textit{state value function} and \textit{action value function}.

\subsection{State Value Function}

\subsection{Action Value Function}

The action value function calculating the expected reward for taking action $a$ in state $s$ under policy $\pi$, is denoted by $q_{\pi}(s, a)$, and defines as
\\
\begin{equation} \label{eq:Action Value Function}
    q_{\pi}(s, a) = \mathbb{E}\Big[G_{t} \Big| S_{t} = s, A_{t} = a\Big] \\
    = \mathbb{E}\Bigg[\sum_{k = 0}^{\infty}\gamma^{k}R_{t + k + 1} \Bigg| S_{t} = s, A_{t} = a \Bigg]
\end{equation}
Where $\gamma$ is the discount rate $0 \leq \gamma \leq 1$, and $\gamma^{k}R_{t + k + 1}$ is the discounted reward at time $t$, after being in state $s$. The discount rate describes the impact of how future rewards influence the return value of the action value function, since $\gamma^{k} \rightarrow 0$, when $k \rightarrow \infty$, unless $\gamma = 1$
\\ \\
By using equation \ref{eq:Action Value Function} we can compute the expected sum of discounted rewards, by taking action $a$ in state $s$ with policy $\pi$. Generally in reinforcement learning we want to maximizing the return value, for the value action function, we want to take action $a$ in state $s$ which maximizing the future discounted reward. We can rewrite equation \ref{eq:Action Value Function}, to a optimal action value function
\begin{equation}
\begin{split}
    q_{*}(s,a) &= \mathbb{E}\Bigg[R_{t + 1} + \gamma  \max\limits_{a^{'}} q_{*}(S_{t + 1}, a^{'}) \Bigg| S_{t} = s, A_{t} = t \Bigg] \\
    &= \sum_{s{'}, r} p(s^{'}, r \Big| s, a) \Big[r + \gamma \max\limits_{a^{'}}q_{*}(s{'}, a^{'})\Big]
\end{split}
\end{equation}
So the optimal action value function return the sum of discounted rewards by taking action $a$ in state $s$. The optimal action value function is recursive, for every element in the sum of the discounted rewards, we compute which action there maximize the discounted reward, until we end up in a terminate state.
\end{document}



%\printbibliography
%\bibliography{citations}
%\bibliographystyle{plain}
\end{document}
