\documentclass[11pt]{article}
\usepackage{mypackages}
\begin{document}

\maketitle

\section{Reinforcement Learning}

Reinforcement Learning (RL) is a machine learning method that focus on learning from interaction.
The basis of reinforcement learning is that given a set of states $S$, that describes the environment,
and a set of actions $A$, that describes how we can interact with the environment,
a \textit{policy} is derived.
A policy is a ruleset that describes which action to take in a state.
The goal of a RL algorithm is to find the policy that
maximizes the \textit{return} of the program.
The return is a reward obtained by a RL algorithm that is usually a number indicating
wether the program performed well or poorly.
Hence the expected return is given by
\begin{equation}
    G_t = \sum\limits_{k=0} \gamma * R_{t}
\end{equation}
where $0 \leq \gamma < 1$ serves as a discount.
The reason for using the discounted return is that the program won't be influenced
too much by events far into the future with big rewards tied to them.

Some problems only award the program for each completed \textit{episode},
while others may choose to asign rewards to every taken action.

An episode starts when the environment is initialised and ends when a terminal state is reached.
For some problems it is difficult to determine if there is a terminal state, since there
is no natural end.

In order to develop a good policy RL methods make use of \textit{exploration} and \textit{exploitation}.
Exploitation is taking the best action based on the expected future return of all available actions.
This means that the program will resort to actions that have previously proved to be profitable.
The problem with only using the greedy approach of exploitation is that the program won't be able
to discover new, and maybe even better, solutions.
Therefore RL algorithms make use of exploration as well by taking a random action based
on a probability distribution.
Usually it serves the program well to start with a high probability of taking random actions
and then reducing it later on, when exploratory moves will most probably only harm the
performance of the algorithm.

\subsection{Value Functions}

Most reinforcement learning algorithms estimates a value function,
that compute how profitable it is to be in a specific state,
or take a specific action given the state.
The term \textit{"profitable"} means what the expected future reward will become.
Generally in reinforcement learning we have two value function
\textit{state value function} and \textit{action value function}.


\documentclass[11pt]{article}
\usepackage{mypackages}
\begin{document}

\maketitle

\subsection{State-value functions}

- The notion of "how good is it to be in this state"
- Defined with respects to policies
- MDP $\rightarrow$ $v_\pi(s)$
- Satisfies a recursive relatonship between current state and "all" successor states
- Optimal value functions?


%\printbibliography
%\bibliography{citations}
%\bibliographystyle{plain}
\end{document}

\documentclass[11pt]{article}
\usepackage{mypackages}
\begin{document}

\maketitle

\subsection{State-value functions}

- The notion of "how good is it to be in this state"
- Defined with respects to policies
- MDP $\rightarrow$ $v_\pi(s)$
- Satisfies a recursive relatonship between current state and "all" successor states
- Optimal value functions?


%\printbibliography
%\bibliography{citations}
%\bibliographystyle{plain}
\end{document}


%\printbibliography
%\bibliography{citations}
%\bibliographystyle{plain}
\end{document}
