\documentclass[11pt]{article}
\usepackage{mypackages}
\begin{document}

\maketitle

\section{Reinforcement Learning}

Reinforcement Learning is a machine learning method that focus on learning from interaction.
The basis of reinforcement learning is that given a set of states $S$, that describes the environment,
and a set of actions $A$, that describes how we can interact with the environment,
a \textit{policy} is derived.
A policy describes which action $a$ to take in which state $s$.
In order to determine wether a policy is good or bad, reinforcement learning uses
rewards and punishments.
Determining to reward or punish the algorithm depends on the problem.
If the problem is simple it is often easier to give rewards, or punishments, based on actions
while it might be better to give more difficult problems only a reward at the end of an \textit{episode}.

An episode starts when the environment is initialised and ends when a terminal state is reached.
For some problems it is difficult to determine if there is a terminal state, since there
is no natural end.

In order to develop a good policy reinforcement learning methods use \textit{exploration} and \textit{exploitation}.

Contrary to supervised learning, where a classifier is build from a set of labeled input and output data,
reinforcement learning focus on i mproving by \textit{exploration} and \textit{exploitation}.

In order to learn how to solve a given problem reinforcement learning needs to be able to
adapt to previously unseen circumstances.
It does this by 


\subsection{Terms}



%\printbibliography
%\bibliography{citations}
%\bibliographystyle{plain}
\end{document}
