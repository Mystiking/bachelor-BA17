\documentclass[11pt]{article}
\usepackage{mypackages}
\begin{document}

\maketitle

\subsection{Q-learning}

An example of learning from the action-value function is \textit{Q-learning}.
The algorithm keeps track of every state-action pair in order to decided which
action is the best to take in a given state.

Q-learning is an off-policy temporal-difference control algorithm, which means
it directly approximates the optimal action-value function, $q_\ast$,
independent of the policy being followed\cite{RLbook}.
The algorithm still uses the policy to decide which state-action pairs are visited
and updated.

This method is important to 

%\printbibliography
%\bibliography{citations}
%\bibliographystyle{plain}
\end{document}
