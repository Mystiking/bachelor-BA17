\documentclass[11pt]{article}
\usepackage{mypackages}
\begin{document}

\maketitle

\subsection{Temporal Difference Learning}

In the previous section we defined the state-value function as
\begin{equation}
    v_\pi(s) = \mathds{E}\bigg[\sum\limits_{k=0}^\infty \gamma^k  R_{t+k+1} \bigg| S_t = s\bigg] \tag{\ref{eq:sv} revisited}
\end{equation}

For most reinforcement learning problems the value of each state
is not known before the learning agent starts experiencing the
effects of its actions.
This means that we will have to estimate the state-value function from this experience, consisting of states, actions taken and the rewards earned
in the transitions from state to state.
We can construct an estimator, $V(S_t) \approx v_\pi(S_t)$, from the experience
since it is a sample of the expected return.
The scheme for updating and maintaining $V(S_t)$ can thus be described as
\begin{equation}
    V(S_t) = V(S_t) + \alpha  [G_t - V(S_t)]
\end{equation}
where $G_t$ is the experienced actual return following from state
$S_t$ and $0 \leq \alpha \leq 1$ is a step-size parameter, 
which defines how much each update is able to change the estimate
in state $S_t$ \cite{RLbook}.

This estimator only allows us to update the estimate, when an epsiode
has ended.
Updating this way is inefficient because it would take the estimator
a lot of episodes to converge towards the true state-value function,
due to it being updated seldomly. 

Instead of updating the estimator $V(S_t)$ when an episode has
ended, we can choose a different approach which updates the $V(S_t)$
for every time step.
We know from equation \ref{eq:sv} that
\begin{equation}\label{eq:one_step}
    \begin{aligned}
        v_\pi(S_t) & = \mathds{E}\bigg[\sum\limits_{k=0}^\infty \gamma^k  R_{t+k+1} \bigg| S_t = s\bigg] \\
                   & = \mathds{E}\bigg[R_{t+1} + \gamma  \sum\limits_{k=0}^\infty \gamma^k  R_{t+k+2} \bigg| S_t = s\bigg] \\
                   & = \mathds{E}\bigg[R_{t+1} + \gamma  v_\pi(S_{t+1}) \bigg| S_t = s\bigg]
    \end{aligned}
\end{equation}

According to \ref{eq:one_step} we can use $R_{t+1}$ instead of $G_t$
and then base the rest of the return on the result of the state-value
function in the next time step, which allows us to estimate $v_\pi(S_t)$ as
\begin{equation}
    V(S_t) = V(S_t) + \alpha  [R_{t+1} + \gamma  V(S_{t+1}) - V(S_t)]
\end{equation}

We refer to this way of updating the estimate as
\textit{Temporal difference learning} (TD) and this particular one-step
update as TD($0$)\cite{RLbook}.
This method uses \textit{bootstrapping}, since we base
part of the estimate on an already existing estimate.

We call the quantity $R_{t+1} + \gamma  V(S_{t+1}) - V(S_t)$
the \textit{TD error}, and denote it by $\delta$.
This error describes the difference in our current prediction of 
$V(S_t)$ and the bootstrapped estimate $R_{t+1} + \gamma 
V(S_{t+1})$.
Generally we want to minimize the TD error, as the function $V(S_t)$
would then be as close to $v_\pi(S_t)$ as possible.
In particular if this error was 0 for all states, then
$V(S_t) = v_\pi(S_t)$ and
\begin{align}
   V(S_t)  & = V(S_t) + \alpha  [R_{t+1} + \gamma  V(S_{t+1}) - V(S_t)]   \\
           & = V(S_t) + \alpha  [V(S_t) - V(S_t)]  \\
           & = V(S_t) 
\end{align}
The one-step estimator of the action-value function from equation
\ref{eq:av} can be derived in a similar fashion.

\subsubsection{Bootstrapping from multiple steps}

The benefit of bootstrapping is that we are able to base part of our estimate on
an already existing estimate.
For some problems bootstrapping over a single time step is useful, since
we are able to update our policy very fast,
but bootstrapping works best if a significant state change has occured
\cite{RLbook}.
If a significant change occurs, it will affect the entire chain
of future estimates from the next visit to those states.
Therefore it is sometimes benificial to update after a period of time
instead of every single time step.
A method that use this multi-step bootstrapping is the \textit{n-step}
TD.

The idea behind n-step TD is to estimate the value function
based on the \textit{n-step return}, $G^{(n)}_t$.
In TD($0$) we estimated the actual return $G_t$ as
$R_{t+1} + \gamma  V_t(S_{t+1})$ for each time step,
where $V_t$ is the estimate of $v_\pi$ at time $t$.
This is also called the one-step return $G^{(1)}_t$.
Extending the one-step return to the n-step return $G^{(n)}_t$ gives us
\begin{equation}
        G^{(n)} = R_{t+1} + \gamma  R_{t+2} + \gamma^2  R_{t+3} + \dots + \gamma^n  V_{t+n-1}(S_{t+n})
\end{equation}
where $n \geq 1$, $0 < t \leq T - n$ and $T - n$ is the nth state before the terminal
state\cite{RLbook}.

In TD($0$) the quantity $R_{t+1} + \gamma  V_t(S_{t+1})$ is
first available in the next time step, and likewise for the n-step TD method future rewards
and value functions aren't available until time $t + n$.
Therefore the natural algorithm for using n-step return to estimate $v_\pi$ is thus
\begin{equation}
    V_{t+n} = V_{t+n-1} + \alpha  [G^{(n)}_t - V_{t+n-1}]
\end{equation}
Since $G^{(n)}_t$ can't be used before $R_{t+n}$ has been experienced
and $V_{t+n-1}$ has been computed, we need to update
$V(S_t)$ for the remaining $n - 1$ following $S_{T-n}$.

\subsubsection{Eligibility traces}

In section \ref{est_vals} we mentioned estimators being represented as
parameterized functions.
Using a parameterized function to estimate the state-value function means
the n-step return can be defined as
\begin{equation}
    G^{(n)} = R_{t+1} + \gamma  R_{t+2} + \gamma^2  R_{t+3} + \dots + \gamma^n  \hat{v}(S_{t+n}, \mathbf{\theta})
\end{equation}
where $\hat{v}(S_{t+n}, \mathbf{\theta}_t)$ is an estimator of the
state-value function with parameters $\mathbf{\theta}_t$.
Instead of keeping track of the last $n$ steps as in the n-step TD
methods,
we can construct a short-term memory vector called an \textit{eligibility trace}, $\mathbf{e}_t \in \R^n$,
that parallels the weight vector $\mathbf{\theta}_t \in \R^n$\cite{RLbook}.
This eligibility trace describes the eligibility of each component in $\mathbf{\theta}_t$ -
how much influence this particular component has in the estimation.
Another advantage of using an eligibility trace is that
learning occurs continually and can affect behaviour immediately,
unlike n-step methods which are always $n$ steps behind.

The eligibility of a component of the weight vector is determined
by its contribution to the most recent computations of the estimate,
where “recent” is determined by the discount rate $0 < \gamma \leq 1$ and the weight
$0 < \lambda \leq 1$.
Thus whenever a component of the weight vector
is used to produce an estimate, the corresponding eligibility component
is increased in either a positive or negative direction and then begins to fade away.
%If a component is rarely used, we provide it with less eligibility
%as it is then less likely to be directly affecting the estimate, and vice versa if
%a component is used often.
This directional change can be found by computing the
gradients of the estimator with respects to $\mathbf{\theta}_t$\cite{RLbook}.
An eligibility trace for $\hat{v}(S_{t+n}, \mathbf{\theta}_t)$ would then
be maintained in the following fashion

\begin{equation}
    \mathbf{e}_t = \nabla(\hat{v}(S_t, \mathbf{\theta}_t)) + \lambda  \gamma  \mathbf{e}_t
\end{equation}
with an initial value of 0 for every component.
The weights of the estimator $\hat{v}(S_t, \mathbf{\theta}_t)$ are
updated every time a state transition occurs, which produces
produces the one-step TD error, $\delta_t = R_{t+1} + \gamma  \hat{v}(S_{t+1}, \mathbf{\theta}_t) - \hat{v}(S_t, \mathbf{\theta}_t)$.
The weights of the estimators we use are updated proportionally to the TD error, since
a low error means we don't have to change much while a large error means that
the estimator needs to be corrected significantly.
A typical update scheme could look like this
\begin{equation}
    \mathbf{\theta}_t = \mathbf{\theta}_t + \eta  \delta_t  \nabla(\hat{v}(S_t, \mathbf{\theta}_t))
\end{equation}
where $0 < \eta \leq 1$ is a step-size parameter limiting how much the weights,
$\mathbf{\theta}_t$, can be changed each time they are updated.


\end{document}
