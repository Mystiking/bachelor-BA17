\documentclass[11pt]{article}
\usepackage{mypackages}
\begin{document}

\section*{Resume}


I dette projekt fremlægger vi effekterne af brugen af asynkron træning
i Deep Reinforcement Learning metoder.
Vi implementerede en Actor-Critic metode med eligibility traces til at løse
CartPole problemet, samt
Asynchronous Advantage Actor-Critic (A3C) algoritmen, som vi brugte
til at løse både CartPole problemet og til at spille Atari spil.
Vores resultater viser at det er muligt at opnå en forøgelse i
antallet af udførte handlinger på 631,27\% i flere Atari spil,
mens stabiliteten af læring vedligeholdes.
Til gengæld, var vi kun i stand til at opnå en formindskelse på 15,09\% for tiden
brugt på at spille Cartpole.
Resultaterne tyder på at fordelene ved asynkron træning kun fremtræder
hvis det er muligt for de enkelte tråde, at udføre nok arbejde mellem hver asynkron opdatering.

\end{document}
