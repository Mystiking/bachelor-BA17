\documentclass[11pt]{article}
\usepackage{mypackages}
\begin{document}

\maketitle


\subsection{Estimating the value of actions}\label{est_vals}

Now that we have introduced the goal of the learning agent as
maximising the total reward, we want to extend this goal to
maximising the reward following a state.
This quantity is called the \textit{return} from state $S_t$ and is
denoted as $G_t$.
The original goal is contained within this new definition, since the total
reward is the return following the starting state $S_0$.
Assuming we receive the rewards $R_{t+1}, R_{t+2}, \dots$ following from state $S_t$
the return can be expressed as the sum of these rewards
\begin{equation}\label{G_t}
    G_t \coloneqq R_{t+1} + R_{t+2} + \hdots + R_{T} = \sum\limits_{i = 1}^T R_{t+i}
\end{equation}
where $T$ is the point in time where the \textit{terminal state} is reached.
The terminal state is the last state in the problem from which
no other states can be reached.
I.e. in backgammon the terminal state is the state where one of the players
have removed all of his pieces from the board and the game is over.
In this project we will only be solving \textit{episodic} problems, which
means there will always be a terminal state.
However, using the sum of rewards as the return of a state proves problematic.
Picture a game of backgammon where a player is given reward 1 for winning the game
and reward 0 for all other actions.
The return of all states will thus be the same
as the value of $R_T$ defines the value of the entire playthrough.
Therefore the learning agent won't be able to tell which actions are the best to take
since all state transitions, except the last, will result in the same return.
To deal with this issue we use the \textit{discounted} sum of rewards as our return instead
of just the sum.
This means that for each time step further into the future
we proportionally discredit the reward.
\begin{equation}\label{gammaG_t}
    G_t \coloneqq R_{t+1} + \gamma R_{t+1} + \gamma^2 R_{t+3} \dots = \sum\limits_{k=0} \gamma^k R_{t+k+1}
\end{equation}
Here, the rate of discount, $ 0 \leq \gamma \leq 1$ determines how far-sighted the learning agent can be,
with a rate of 0 meaning the return is only the most recent reward and a rate of 1 giving us equation \ref{G_t}.


Now that the return has been defined, lets consider the response from the environment at time $t + 1$
to the action taken at time $t$.
According to the agent-environment model (Fig. \ref{fig:agent_environment}) this response consists
of a new state $S_{t+1} = s'$ and a corresponding reward $R_{t+1} = r$.
Assuming the response depends on all past actions $A_{0}, A_{1}, \cdots, A_{t-1}$ and
corresponding responses $S_{1}, R_{1}, \cdots, S_{t}, R_{t}$, the probability
of arriving in state $S_{t+1}$ and receiving reward $R_{t+1}$ can be defined as the
joint probability given by
\begin{equation}
    \mathds{P}(S_{t+1} = s', R_{t+1} = r | S_0, A_0, R_1, \cdots, S_{t-1}, A_{t-1}, R_{t}, S_{t}, A_{t})
\end{equation}
where $S_0$ is the inital state and $A_0$ the first action taken.

This notation is tedious so to avoid it we will assume that all Reinforcement Learning tasks
we encounter have the \textit{Markov property}.
A task has this property if a state can summarize everything important that
has happened up until this point in time.
In the Cart-Pole and Atari games the Markov property is present because each state provides
an amount of information that makes previous states unnecessary.
This property allows us to describe the joint probability for reaching state $s'$ and
experiencing reward $r$ from state $s$ performing action $a$ as
\begin{equation}\label{joint_prob}
    \mathds{P}(S_{t+1} = s', R = r | S_t = s, A_t = a)
\end{equation}
since $S_t$ retains all information from states $S_0, S_1, \dots, S_{t-1}$
which also means we only have to concern ourseves with the current action $A_t$.

The actions taken are sampled from the probability distribution $\pi$, which means
we can't truly know which states and rewards we will encounter.
Therefore the return $G_t$ we defined in equation \ref{gammaG_t}
can only be used to describe the return from state $S_t$ if 
the policy is deterministic.
In this project we will not be using deterministic policies, so 
instead we use the expectation of the return to value how
good it is to be in a state.
We denote finding the value of a state $s$ following policy $\pi$ as
$v_\pi(s)$ - a \textit{state-value function}.

Now, for every action in every state there is
probability $p(s', r | s, a)$ of arriving at exactly state $s'$ and
receiving reward $r$.
This means that for each action we need to go through all possible future
states and rewards, giving us probability $\pi(a|s)p(s',r|s, a)$ of receiving
exactly reward $r$ and encountering future state $s'$ from state $s$ taking
action $a$.
Therefore the value of taking action $a$ in state $s$ is the expected
return of all possible future states $s'$ and corresponding rewards $r$.
Finding the value of an action can thus be defined as
\begin{equation}
    q_\pi(s, a) = \sum\limits_{s', r} p(s', r | s, a) [r + v_\pi(s')]
\end{equation}
and is refered to as the \textit{action-value function}.

To extend this idea to the value of a state, we simply have to take the expectation over
every possible action and hence
\begin{equation}\label{eq:sv}
    v_\pi(s) = \sum\limits_{a}\pi(a|s) \sum\limits_{s', r} p(s', r | s, a) [r + v_\pi(s')]
\end{equation}
which means we the value function is defined recursively in terms of its future estimate until the
terminal state is reached, where the value always has to be 0 since no more rewards can be obtained
from this point.

%\printbibliography
%\bibliography{citations}
%\bibliographystyle{plain}
\end{document}
