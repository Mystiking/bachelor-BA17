\documentclass[11pt]{article}
\usepackage{mypackages}
\begin{document}

\maketitle

\subsection{State-value functions}

One way to find an optimal policy is to maximise the \textit{state-value} function.
This function is used to estimate the reward that can be expected given a state $s$.
As described in Section \ref{policy} a policy is a mapping from each state $s \in \mathcal{S}$, and
action $a \in \mathcal{A}(s)$ to the probaility $\pi(a|s)$ of taking action $a$ in state $s$.
The value of the state $s$ is thus defined as the expected return when the policy
$\pi$ is followed from state $s$.
We denote the value function for a state $s$ as
\begin{equation}\label{eqstatevalue}
    v_\pi(s) \doteq \mathds{E}[G_t | S_t = s] =
    \mathds{E}
    \left [
        \sum\limits_{k=0}^\infty \gamma^k R_{t + k + 1} | S_t = s
    \right ]
\end{equation}
over the policy $\pi$ where $G_t$ is the sum of discounted rewards following time $t$.

In order to estimate the value function $v_\pi$ the agent has to experience
the environment.
Since state-value functions depend on maintaining and evaluating the values of all states
it becomes infeasable to keep track when the amount of states is high or the
environment is continous.
Therefore the state-value function is often approximated by a parameterized function
with less parameters than states\cite{RLbook}.
In deep reinforcement learning the approximator can be a neural network
and this approach will be discussed in detail in a later chapter.

Learning from experience can be expressed as a recursive relationship where the
value of the current state is based on the value of possible future states.
The relationship is shown in equation \ref{eq312}.
\begin{equation}\label{eq312}
    \begin{split}
        v_\pi(s) & \doteq \sum\limits_a \pi(a|s) \sum\limits_{s',r}p(s',r|s, a)\left [r + \gamma v_\pi(s') \right ]
    \end{split}
\end{equation}
 
The value of a state is given by the reward of entering the state, $r$, and the discounted value of the successor state
$\gamma v_\pi(s')$.
This value depends on the probability of entering state $s'$ and recieveing reward $r$ given
the starting state and an action $a$ - $p(s', r | s, a)$.
Since the action $a$ is taken with probability $\pi(a|s)$ the probability of taking an action and ending in state $s'$ with reward $r$
can be expressed as $\pi(a|s) p(s', r | s, a)$.
The value of taking action $a$ given state $s$ can be estimated as $\sum\limits_{s',r} p(s',r|s,a)\left [r + \gamma v_\pi(s') \right ]$.
To estimate the value of the entire state we need to take all actions into account, or more specifically,
the probability of taking the actions.
This means we can describe the estimation of a state as
$
v_\pi(s)  \doteq \sum\limits_a \pi(a|s) \sum\limits_{s',r}p(s',r|s, a)\left [r + \gamma v_\pi(s') \right ]
$.

The equation given above and in equation \ref{eq312} describes the same as equation \ref{eqstatevalue} since

%%% Equation from 63 RL book

- The notion of "how good is it to be in this state"
- Defined with respects to policies
- MDP $\rightarrow$ $v_\pi(s)$
- Satisfies a recursive relatonship between current state and "all" successor states
- Optimal value functions?


%\printbibliography
%\bibliography{citations}
%\bibliographystyle{plain}
\end{document}
