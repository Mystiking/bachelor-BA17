\documentclass[11pt]{article}
\usepackage{mypackages}
\begin{document}

\section{Discussion and Further Work}

It seems evident that the methods used in this project were able
to solve the CartPole problem and several Atari games, with the exception of Pong, successfully.
The reason we weren't able to learn how to play Pong
may be rooted in the preprocessing step.
The preprocessing computes the mean of the surrounding pixels
to represent the three channels of the RGB Atari frames as a single
value for each pixel.
For both Space Invaders and Breakout, most of the background pixels are
black, which means a lot of the pixel values in the grayscaled images
are zero.
When we resize these images, a lot of the zero values are retained,
and when they are fed to the neural network, they won't have any
influence on the final outcome.
For Pong, this is not the case since the background is orange.
This means that the input is less sparse than for the other games,
which can be the reason that the A3C method is unable to learn
how to play the game.
However, we believe that the method would be able to converge, if
it had been given more time to train
since there is no theoretical reason it shouldn't,
but due to the time limitations
of this project, we haven't been able to test it.

In the CartPole experiments, it is notable that the A3C method converged
towards a mean score of 200, faster than the Actor-Critic method with
eligibility traces for all thread settings.
A possible reason for the difference in performance is that we were able to
take larger gradient steps using the A3C method
because a large step-size in the Actor-Critic method 
resulted in an exception.

From figure \ref{fig:a3c_time_steps} and \ref{fig:a3c_time} we saw 
that there was no significant speedup in the improvement of performance by
using multiple threads in the CartPole experiments.
We expected the speed-up in regards to real time to be more significant
since we should be able to perform a lot more time steps
by running multiple threads in parallel. 
During training, we updated the global model after each timestep,
such that the updating scheme of the A3C method was more like the one
used in the Actor-Critic method.
The lack of speed-up could be a consequence of updating the network requiring
approximately the same amount of time, as it took to sample and perform
an action.
Hence, the local agents would be waiting to
perform the asynchronous update of the global model, which means the addition of
more threads, simply results in a larger queue.

The results of the Atari games behaved a lot more as expected.
More threads resulted in a higher amount of completed timesteps between each thread setting,
which suggests that the bottleneck from the CartPole experiment isn't present in
the A3C implementation for the Atari games.
For this method, we used an action repeat, as well as a replay memory,
which means we spent more time interacting with the environment
between the asynchronous updates.
From figure \ref{fig:a3c_spaceinvaders} to \ref{fig:a3c_pong_comp}
it is clear that more threads results in a larger amount of completed timesteps
in the same amount of time,
however, it is notable that it seems like all the different thread
settings converge in the same amount of timesteps.
This means that the A3C algorithm is stable since the learning
doesn't collapse, even though multiple agents update the global parameters asynchronously 
while being situated in vastly different states.

%%In \cite{a3c} multiple different optimization schemes for performing
%%the aynchronous update was discussed, and presented results showing
%%RMSProp with shared statistics performed the best.
%%We have chosen to test our implementation using the RMSPropagation
%%with shared statistics as optimizer, due to the reasons discussed in
%%section \ref{sec:a3c} and the results presented in the article.
%%This optimzer is part of the reason the A3C experiments didn't collapse

For the Atari games, it is unexpected that the number of completed timesteps
for two, four and eight threads are very close to each other, but
sixteen threads complete 145\% more timesteps than the experiment using eight
threads, on average over all of the games.
All of the Atari experiments were performed on DIKUs GPU servers since each of them
has 24 CPU cores.
A problem with this setting is that we can't be sure 
the experiments were the only processes running on the servers, which means
some of the threads might have to wait for other users to
give up resources.
This would explain why there is a large gap between the timesteps
completed using eight and sixteen threads, due to one
being luckier with regards to user activity.
Another problem is that the experiments have been performed on servers with different CPUs,
which means that there is no guarantee that the difference isn't due to
varying hardware quality.

Unlike the CartPole experiments, we have only run our implementation
of A3C for the Atari games a single time with each thread setting for each game.
Therefore the gap in speed-up might even out if the experiments were allowed to run
several times since we believe some of the experiments could have been influenced
by outside factors.
However, for all the Atari games we achieved roughly the same mean score
for all thread settings after the same amount of timesteps,
which seems to imply that the learning is stable.
Each model was initialized
with the same parameters, and the results indicate that running an experiment
an additional time, for the same amount of timesteps,
would yield roughly the same result.

The result from the experiments for Breakout and Space Invaders
seems to indicate, that once the method reaches a certain point
the improvement in performance becomes stale.
In most Atari games the level of difficulty increases as the
player progresses through the game.
In Breakout the speed of the ball increases after
four hits and again after twelve.
This explains why the increase in performance becomes lower after
approximately half a million timesteps and again after
roughly 2.5 million - the points in time where a mean score around 4 and 12
is consistently produced.
For Space Invaders the same scenario is present since
the aliens progressively increase their speed as they
descend upon the player.
This means that it is more difficult to hit the aliens towards the end of an episode,
which explains the results from figure \ref{fig:a3c_spaceinvaders}
since the mean score is just short of the score obtained by killing all 36 aliens (630 points).

Overall,
it seems that the bottleneck of the positive effects of asynchronous training
lies in the amount of work that can be done in between updates to the global model.
Hence, multiple threads only increase the pace of learning, if
interacting with the environment takes long enough to ensure
none of the threads will have to wait to perform
its asynchronous update.
It would be interesting to investigate where the threshold is located
for the speed up in regards to the number of threads used in the Atari games,
and if we have reached it already, but we will have to leave this for future work.
Another possible limitation might lie in the size of the replay memory.
In the article, \cite{a3c}, no experiments are provided which test
the effects of different sizes.
It would be interesting to examine these effects since a larger replay memory,
in theory, should make it possible to use more threads, but at the same time,
the accumulated gradient updates will become larger, which might result in
more variance in the final result.

In this project, we have used Deep Reinforcement Learning to
create a model capable of playing Atari games.
However, the concepts and ideas discussed can be translated to real-world
issues as well.
The advantage of learning from experience can be used to solve problems where the
domain is either too large or too complex to
describe with a set of rules.
A possible application for Deep Reinforcement Learning is
to use it as the controller of self-driving cars.
The states of the model would, in this case, be information
from the real world, e.g. images of the road ahead, 
and the actions would be the same as those available to
a human driver.
The Reinforcement Learning agent would, in this case, be able to pick the
optimal action to perform in a real-world situation since
it has experienced a lot of similar states previously.
However, when unique and unfamiliar situations occur 
the agent needs to select the best action based on limited to no
experience.
Therefore Deep Reinforcement Learning can only be applied
to real world issues if a “safety net” is in place
to handle critical situations that don't occur
very often in the training, as described in \cite{BILER}.

\end{document}
